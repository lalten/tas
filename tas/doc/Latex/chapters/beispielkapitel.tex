%
% Beispielkapitel: Bedienung dieser Vorlage
%

% Zum Setzen von TeX-Quellcode, der nicht interpretiert werden soll
\newcommand{\makro}[1]{\texttt{\textbackslash{}#1\{\}}}

\chapter{Unser erstes Beispiel}
\label{sec:Beispiel}

Dies ist das erste Beispiel-Kapitel, in dem einige grundlegende Funktionen dieser \LaTeX{}-Vorlage erklärt werden.
In der richtigen Arbeit beginnt hier der eigentliche Bericht.

%=========================================================================================
\section{Struktur}
%=========================================================================================
\label{sec:Struktur}

In diesem Absatz steht nicht so viel, aber immerhin besteht er aus mehreren Sätzen.
Es ist günstig, im Quellcode nach jedem Satz einen Zeilenumbruch zu setzen.

Im Haupttext erscheint ein Zeilenumbruch erst, wenn im Quelltext eine Leerzeile auftritt.

Jedes Kapitel sollte eine eigene Quellcode-Datei erhalten, die sich im Ka\-pi\-tel-Un\-ter\-ver\-zeich\-nis befindet.
Diese Dateien werden in der Hauptdatei \texttt{bericht.tex} mit dem \makro{include}-Befehl eingebunden.
Es bietet sich an, sections und subsections im Quellcode durch Kommentarzeilen hervorzuheben.

%=========================================================================================
\subsection{Test einer strukturierten Gliederung}
\label{sec:GliederungsTest}
Hier gehts in eine Ebene tiefer weiter. 
Im Rahmen einer Studien- oder Diplomarbeit sollten normalerweise die ersten drei Gliederungsebenen vorkommen (\makro{chapter}, \makro{section} und \makro{subsection}).

%=========================================================================================
\section{TODO-Liste}
%=========================================================================================
Wenn man mit irgendetwas nicht fertig geworden ist, kann man sich mit dem \makro{todo}-Makro eine Anmerkung machen.
Diese Anmerkungen werden am Ende des Dokumentes aufgelistet, sofern es noch welche gibt.
\todo{Noch mehr schreiben}

%=========================================================================================
\section{Verlinkungstest}
%=========================================================================================
\label{sec:Verlinkungstest}

Hier prüfen wir die Verlinkung, z.\,B. zum \refsec{GliederungsTest}.
Vorausgesetzt, alle Kapitellabel beginnen ordentlich mit \texttt{sec:}, kann man das \makro{refsec}-Makro benutzen und braucht nicht jedes Mal Kapitel zu schreiben.
Labels werden übrigens nicht automatisch erstellt, das muss man von Hand machen.
Nicht vergessen, dass ein Dokument zweimal erstellt werden muss, um die Verlinkung
ordentlich hinzukriegen!

%=========================================================================================
\section{Gleichungen}
%=========================================================================================
Gleichungen werden mit der \texttt{equation}-Umgebung erstellt.
Dann werden sie auch direkt ordentlich numeriert, wie z.\,B. \refeq{Fundamentalgleichung}
Für mehrere gestaffelte Gleichungen gibt es die \texttt{align}-Umgebung.
Die doppelten Dollarzeichen \$\$ sollten vermieden werden.
\begin{equation}
	e^{i \pi} + 1 = 0
	\label{eq:Fundamentalgleichung}
\end{equation}
\nomenclature[Ae]{$e$}{Eulersche Zahl}{--}{skalar}
\nomenclature[G]{$\pi$}{Kreiszahl}{--}{skalar}
\nomenclature[X]{$k$}{Zeitschritt}{--}{skalar} %ok, kommt in der Gleichung nicht vor, nur zur Demo

Matrizen und Vektoren sollten mit den zugehörigen \makro{vec}- und \makro{mat}-Makros (definiert in \texttt{makros.tex}) gekennzeichnet werden, um die Schreibweise zentral anpassen zu können.
Matrizen sollten Großbuchstaben, Vektoren Kleinbuchstaben als Symbole erhalten.
\begin{equation}
	\mat{A} \vec{x} = \vec{b}
	\label{eq:LGS}
\end{equation}
\nomenclature[AA]{$\mat{A}$}{Systemmatrix}{div.}{$4 \times 4$}
\nomenclature[Ax]{$\vec{x}$}{Vektor der Unbekannten}{div.}{$4 \times 1$}

Es ist auch sinnvoll, sich in \texttt{makros.tex} Abkürzungen für häufig verwendete mathematische Schreibweisen, Symbole oder ähnliches zu definieren.
Sollte eines dieser Symbole einmal geändert werden müssen, kann das dann zentral geschehen.

Zahlen und Einheiten werden am besten mit der \texttt{siunitx}-Package gesetzt.
Zahlen können so auch in Exponentenschreibweise eingegeben werden, z.\,B. 6.674e-11 = \num{6.674e-11}, Einheiten werden automatisch richtig und einheitlich gesetzt.
Das geht sowohl im Fließtext ($g=\SI{9.81}{m/s^2}$), als auch in abgesetzten Gleichungen:
\begin{equation}
	g = \SI{9.81}{\metre\per\second\squared}
	\label{eq:g}
\end{equation}
\nomenclature[Ag]{$g$}{Erdbeschleunigung}{\si{\metre\per\second\squared}}{skalar}

Formelzeichen sollten direkt bei ihrer ersten Verwendung in einer Formel mit dem \texttt{\textbackslash{}nomenclature}-Befehl definiert werden.
Sie erscheinen dann automatisch im Symbolverzeichnis.
Das gilt auch für Aküs!
\nomenclature[S]{Aküs}{Abkürzungen}{}{}

Der \texttt{\textbackslash{}nomenclature}-Befehl erhält als erstes (optionales) Argument die Untergruppe (s.~\vreftab{nomenclArgumente}), dann das Formelzeichen, Beschreibung, Einheit und Dimension.
Die Untergruppe kann um einen weiteren Buchstaben erweitert werden, der dann die (alphabetische) Reihenfolge in der Liste bestimmt.
Das sieht im Quelltext dann etwa so aus:
\begin{verbatim}
\nomenclature[AA]{$\mat{A}$}{Systemmatrix}{div.}{$4 \times 4$}
\end{verbatim}

Nicht vergessen, dass der entsprechende makeindex-Befehl ausgeführt wird!
Dazu in TeXnicCenter Ausgabe$\rightarrow$Ausgabeprofile definieren wählen, das Profil LaTeX => PDF kopieren nach LaTeX => PDF mit nomentbl und als Argument für makeindex \texttt{\%tm.nlo -s nomentbl.ist -o \%tm.nls} eingeben.
Außerdem muss in den Projekteigenschaften der Haken bei \glqq{}Verwendet makeindex\grqq{} gesetzt sein.

\begin{table}[htbp]
	\centering
		\begin{tabular}{ll}
			\hline
			A & Lateinische Buchstaben \\
			G & Griechische Buchstaben \\
			S & Abkürzungen \\
			X & hochgestellte Indizes \\
			Z & tiefgestellte Indizes \\
			\hline
		\end{tabular}
	\caption{Argumente für die Untergruppe des \texttt{\textbackslash{}nomenclature}-Befehls}
	\label{tab:nomenclArgumente}
\end{table}

%=========================================================================================
\section{Tabellen und Abbildungen}
%=========================================================================================
\label{sec:TabellenUndAbbildungen}

Hier gibt's ein Beispiel für schöne Tabellen und Bilder.
Alle Tabellen und Bilder sollten unbedingt im Text referenziert werden, wie z.\,B. \vreffig{Waagenstiel1} hier!
Bilder können als pdf, jpg oder png eingebunden werden.
Die Dateien werden im Unterverzeichnis Bilder abgelegt.
Für die Verlinkung gibt es auch entsprechende Makros, nämlich \makro{reffig} und \makro{reftab}.
Gerade bei Abbildungen, die häufig auf andere Seiten rutschen, bietet sich auch die Variante \makro{vreffig} an, die ggf. noch einen Verweis anfügt, wo genau die Abbildung sich befindet. 
Diese Variante gibt es natürlich auch für Kapitel, Gleichungen und Tabellen.

Tabellen und Abbildungen werden auch automatisch ins entsprechende Verzeichnis aufgenommen.
Gerade bei den Linien der Tabellen gilt übrigens, dass weniger oft mehr ist!
Eine Linie drüber und eine drunter reicht oft aus, wie z.\,B. in  \vreftab{MagischesQuadrat}.
Hat die Tabelle eine Überschrift, kann diese durch eine dritte Linie abgegrenzt werden.

\begin{table}[htbp]
	\centering	
	\begin{tabular}{ccc} % eine Tabelle mit drei Spalten, in denen der Text jeweils zentriert ist
		\hline 
		8 & 1 & 6 \\
  	3 & 5 & 7 \\
  	4 & 9 & 2 \\
		\hline
	\end{tabular}
	\caption{Magisches Quadrat} % Die Beschriftung der Tabelle
	\label{tab:MagischesQuadrat}
\end{table}

\begin{figure}[htbp]
	\centering
		\includegraphics{Waagenstiel1}
	\caption{Der Stiel der Windkanalwaage}
	\label{fig:Waagenstiel1}
\end{figure}

%=========================================================================================
\section{Literaturverweise}
%=========================================================================================
Literaturverweise werden zentral in einer Bibtex-Datei (hier heißt sie \texttt{literatur.bib}) gesammelt.
Diese Datei kann bequem mit Tools wie etwa JabRef \cite{JabRefURL} verwaltet werden.
Im Text werden die entsprechenden Referenzen mit dem \makro{cite}-Befehl eingesetzt, das Literaturverzeichnis wird dann automatisch erstellt.
Werden Bücher oder sonstige längere Abhandlungen zitiert, so sollte auf die eigentliche Referenz auch eine Seiten- oder Kapitelangabe folgen, wie z.\,B. \cite{Oetiker2008}, Kap.~4.2.
Mit dem \makro{nocite}-Befehl in der Datei \texttt{bericht.tex} können außerdem unzitierte Quellen hinzugefügt werden, wie es hier mit einer Reihe von \LaTeX{}-Anleitungen geschehen ist.
