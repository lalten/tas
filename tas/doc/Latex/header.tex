%====================================================================================================================
% Headerdatei f�r einen Bericht
%====================================================================================================================

%====================================================================================================================
% Die KOMA-Skript Dokumentklasse "scrreprt" verwenden

\documentclass[
		english,
		% eine der beiden folgenden Optionen ausw�hlen:
		final,				% hiermit wird die endg�ltige Fassung des Dokuments erstellt
%		draft,				% hiermit wird eine Rohfassung erstellt, 
									% in der die Bilder nur als Rahmen erscheinen und schwarze Balken neben
									% �bervolle Boxen gezeichnet werden
		pdftex,				% automatische PDF-Verlinkung, bessere Bildereinbindung
		12pt,					% Standardschriftgr��e
		fleqn,				% Formeln linksb�ndig
		a4paper,			% DIN-A4-Papier
		titlepage,		% extra Titelseite verwenden
		numbers=noenddot, % �berschriften ohne Punkt (nach DIN eigentlich mat, wenn auch Buchstaben verwendet werden)
		bibliography=totoc			% Literaturverzeichnis im Inhaltsverzeichnis
		%chapterprefix	% "Kapitel x" zus�tzlich
]{scrreprt}
\pdfoptionpdfminorversion=5 % akzeptiere auch Bilder in PDF Version 1.5

%====================================================================================================================
% Pakete laden

% Seitenr�nder und Kopf-/Fu�zeilen einstellen
\usepackage[a4paper,
	headsep=0.5cm,
	footskip=0.8cm,
	left=3.5cm, right=2.5cm,
	top=2.5cm, bottom=2.0cm]{geometry}
% TEST: Blattgeometrie auf der ersten Seite anzeigen
%\geometry{showframe}

% Intelligente Verweise (mit automatischer Erg�nzung von "auf Seite XX")
% Achtung, nicht nach hyperref laden!!!
% wird aktiviert, falls \vref statt \ref benutzt wird
\usepackage{varioref}

% Kopf- und Fu�zeilen ver�ndern
\usepackage{fancyhdr}
\pagestyle{fancy}
% Kopfzeile: Lehrstuhlsschriftzug links, Logo rechts
\lhead{\sffamily{\Large Lehrstuhl f�r Steuerungs- und Regelungstechnik\\ \small Technische Universit�t M�nchen}}
\rhead{\includegraphics[height=1.25cm]{pics/lsr_logo.png}}
\setlength{\headheight}{40pt}
% erh�hten Platz f�r Kopfzeile ber�cksichtigen
\addtolength{\textheight}{12pt}         % alte H�he dazu
\addtolength{\textheight}{-\headheight} % und neue wieder abziehen

% Folgende Befehle einkommentieren, um Abs�tze durch Abst�nde zu trennen
% Werden sie auskommentiert, so beginnt jeder Absatz mit einem Erstzeileneinzug
% Einzug bei Absatzbeginn
\parindent=0cm
% Abstand zwischen Abs�tzen (hier gleich Zeilenabstand)
\parskip=\baselineskip

% Deutsche Trennungen, Anf�hrungsstriche und mehr:
\usepackage[german english]{babel}
%\usepackage{english}

% Umlaute direkt eingeben (�,�,�)
% Auf Linuxsystemen hier eher utf-8 statt latin1
\usepackage[utf8]{inputenc}

% intern: � gleich als ganzes Zeichen darstellen und nicht aus ^ und a zusammensetzen
\usepackage[T1]{fontenc}

% Zum Einbinden von Graphiken
\usepackage{graphicx}
\graphicspath{{pics/}}

% Mathematische Symbole f�r
\usepackage{amsmath}
\usepackage{amssymb}

% Automatische Erstellung eines Symbolverzeichnisses
% Dazu muss ein spezieller MakeIndex-Lauf nach jedem Erstellen erfolgen, 
% siehe Dokumentation zu nomencl und nomentbl
\usepackage[german]{nomentbl}
\makenomenclature

% Deutsche BiBTeX-Styles nach DIN 1505
\usepackage{bibgerm}
% \bibliographystyle{alphadin}          % alphabetische K�rzel
\bibliographystyle{plaindin}				% numerische K�rzel

% verlinktes Inhaltsverzeichnis, logische Seitenzahlen im PDF etc.
\usepackage[
	plainpages=false, 
	pdfpagelabels, 
	colorlinks=false, % true: Links werden farbig gedruckt (Bildschirm-Version) 
	                  % false: Links werden schwarz gedruckt, haben aber bunte Box auf dem Bildschirm (Druckversion)
	linkcolor=blue,   % Farbe f�r interne Links
	final]
	{hyperref}
	
% Pakete von Sch�lls Beispiel
\usepackage{longtable}            % F�r evtl. mehrseitige Tabellen
\RequirePackage{array}
\usepackage{dcolumn}              % in Tabellen an Dezimaltrennzeichen ausrichten
\newcolumntype{.}[1]{D{.}{.}{-1}} % neuer Spaltentyp: .{} richtet am Dezimalpunkt zentriert aus
\newcolumntype{d}[1]{D{.}{.}{#1}} % neuer Spaltentyp: d{x} richtet am Dezimalpunkt mit x Nachkommastellen aus

\usepackage{subfig}               % F�r mehrteilige Abbildungen
\usepackage{textcomp}             % F�r diverse Sonderzeichen, u.a. \textdegree f�r Gradzeichen
\usepackage{booktabs}             % F�r schicke Linien in z.B. Tabellen
\usepackage[subfigure]{tocloft}   % Erstellen eigener Verzeichnisse, ben�tigt f�r das TODO-Makro
\usepackage{ifthen}               % Entscheidungen im Quellcode
\usepackage{color}                % in Farbe uund buuuunt!

% Korrektes Setzen von Ma�einheiten, Zahlen u. a.
\usepackage[per-mode=symbol]{siunitx}

% Ein paar Einstellungen f�r das zu produzierende PDF
\hypersetup{
  pdfproducer={\pdftexbanner},
  bookmarksnumbered=true,
  bookmarksopen=false,
}

\usepackage{listings}
\usepackage[font={small,color=black}, labelfont=bf,figurename=Abb.]{caption} 
\usepackage{pxfonts}

\usepackage{pgfplots}
\usepackage{filecontents}
\usepackage[miktex]{gnuplottex}

\usepackage{wrapfig}


\newcommand{\shellcmd}[1]{\indent\indent\texttt{\scriptsize\# #1}}